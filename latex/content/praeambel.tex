% ju 2-Mrz-20 praeambel.tex
\RequirePackage[l2tabu,orthodox]{nag}    % Detecting and warning about obsolete LaTeX commands
\RequirePackage[T1]{fontenc}      % Standard package for selecting font encodings
\RequirePackage{textcomp}         % LaTeX support for the Text Companion fonts
\RequirePackage[utf8]{inputenc}   % Accept different input encodings
\RequirePackage[dvipsnames,usenames]{xcolor}    % Driver-independent color extensions for LaTeX and pdfLaTeX

\RequirePackage[osf,sc]{mathpazo} % Fonts to typeset mathematics to match Palatino
\RequirePackage[scale=.9,semibold]{sourcecodepro}    % Use SourceCodePro with TeX(-alike) systems
\RequirePackage[osf]{sourcesanspro}    % Use SourceSansPro with TeX(-alike) systems

%\usepackage{lmodern}
%\usepackage{eulervm}
\usepackage[ngerman]{babel}% neuen deutschen Rechtschreibung
\usepackage[%
autostyle=true,% Anpassung an babelmain-
%english=american,%
german=guillemets% Design der deutschen Anfuehrungszeichen
]{csquotes}% praktische Anfuehrungszeichen

% bibliography
\usepackage[
bibencoding=utf8,
backend=bibtex,% bibtex, biber
sorting=nty,%Sortierung nach Name, Title, Year
style=alphabetic-verb,%beim Zitieren: alphabetic-verb, numeric, mla, science, phys, nature, ieee
%bibstyle=alphabetic%im Verzeichnis  : alphabetic, numeric, authoryear 
backref=false,backrefstyle=three+,url=true,urldate=comp,abbreviate=false,maxnames=20
]{biblatex} %Paket laden


\RequirePackage{amsmath}          % AMS mathematical facilities for LaTeX
\RequirePackage{array}            % Extending the array and tabular environments
\RequirePackage{chngcntr}         % Change the resetting of counters


\RequirePackage{csquotes}         % Context sensitive quotation facilities
\RequirePackage{colortbl}         % Add colour to LaTeX tables
\RequirePackage{etoolbox}         % e-TeX tools for LaTeX
\RequirePackage{enumitem}         % Control layout of itemize, enumerate, description
\RequirePackage{float}            % Improved interface for floating objects
\RequirePackage{footnote}         % Improve on LaTeX's footnote handling
\RequirePackage{fnpct}\setfnpct{after-punct-space={-.2em}}    % Manage footnote marks’ interaction with punctuation
\RequirePackage{graphicx}         % Enhanced support for graphics
\RequirePackage{listings}         % Typeset source code listings using LaTeX

\RequirePackage{caption}          % Customising captions in floating environments
\RequirePackage{makeidx}          % Standard LaTeX package for creating indexes
\RequirePackage{multirow}         % Create tabular cells spanning multiple rows
\RequirePackage{scrhack}          % KOMA-Script File, contains improvement proposals for other packages
\RequirePackage[stretch=9,shrink=15,step=3,tracking=smallcaps,letterspace=75,final,babel=true]{microtype}    % Subliminal refinements towards typographical perfection
%\RequirePackage[osf,sc]{mathpazo} % Fonts to typeset mathematics to match Palatino
%\RequirePackage[scale=.9,semibold]{sourcecodepro}    % Use SourceCodePro with TeX(-alike) systems
%\RequirePackage[osf]{sourcesanspro}    % Use SourceSansPro with TeX(-alike) systems
\RequirePackage{textcase}         % Case conversion ignoring mathematics, etc
\RequirePackage{tikz}             % Create PostScript and PDF graphics in TeX
\RequirePackage{tocbasic}         % Management of tables/lists of contents (and the like)
\RequirePackage{todonotes}        % Marking things to do in a LaTeX document

\usepackage{calc}
\usepackage{setspace}

\usepackage{subcaption}

\usepackage{color}
\usepackage{multicol}
\usepackage{framed}
\usepackage[most]{tcolorbox}
\usepackage{wrapfig}

\usepackage{verbatim}
\usepackage{colortbl}
\usepackage{array, booktabs, caption}
\usepackage{makecell}
\usepackage{tcolorbox}
\usepackage{lipsum}
\usepackage{microtype}
\usepackage{pdfpages}% PDF Datei einbinden

% eigene Farbe definieren
\definecolor{farbe1}{gray}{0.5}
\definecolor{farbe2}{rgb}{1,0.4,0.3}
\definecolor{farbe3}{RGB}{255,60,40}
\definecolor{farbe4}{HTML}{FF00AA}
% Adobe Prozessfarben: CMYK: 100,50,0,35 -> 1,0.5,0,0.35
% Anwendung: \colorbox{blau}{Box}
\definecolor{blau}{cmyk}{1,0.5,0,0.35}     % 100,50,0,35
\definecolor{schwarz}{cmyk}{0.4,0.2,0.2,1} % 40,20,20,100
\definecolor{rot}{cmyk}{0,1,1,0.1}         % 0,100,100,10
\definecolor{orange}{cmyk}{0,0.55,0.61,0}     % 0,55,61,0
\definecolor{rot1}{cmyk}{0,0.95,0.31,0}       % 0,95,31,0
\definecolor{rot2}{cmyk}{0.13,1,0.4,0.04}     % 13,100,40,4
\definecolor{rot3}{cmyk}{0.29,1,0.47,0.34}  % 29,100,47,34
\definecolor{rot4}{cmyk}{0,0.95,0.9,0}        % 0,95,90,0
\definecolor{rot5}{cmyk}{0.22,1,1,0.19}   % 22,100,100,19
\definecolor{rot6}{cmyk}{0.33,1,0.95,0.05}   % 33,100,95,5
\definecolor{blau1}{cmyk}{0.71,0.15,0,0}      % 71,15,0,0
\definecolor{blau2}{cmyk}{0.85,0.42,0.18,0.04} % 85,42,18,4
\definecolor{blau3}{cmyk}{0.96,0.66,0.45,0.44}% 96,66,45,44
\definecolor{blau4}{cmyk}{0.83,0.56,0,0}      % 83,56,0,0
\definecolor{blau5}{cmyk}{1,0.77,0.1,0.01}  % 100,77,10,1
\definecolor{blau6}{cmyk}{1,0.86,0.4,0.35}  % 100,86,40,35
\definecolor{grau1}{cmyk}{0,0,0,0.2}          % 0,0,0,20
\definecolor{grau2}{cmyk}{0,0,0,0.4}          % 0,0,0,40
\definecolor{grau3}{cmyk}{0,0,0,0.8}          % 0,0,0,80
\definecolor{lila1}{cmyk}{0.36,0.67,0,0}      % 36,67,0,0
\definecolor{lila2}{cmyk}{0.45,0.8,0,0}       % 45,80,0,0
\definecolor{lila3}{cmyk}{0.73,1,0.26,0.17}   % 73,100,26,17
% hell - info
\definecolor{blau-hell}{cmyk}{0.25,0.13,0,0}     % 25,13,0,0
\definecolor{blau-dunkel}{cmyk}{0.75,0.45,0.05,0}   % 75,45,5,0
\definecolor{gruen-hell}{cmyk}{0.12,0,0.24,0}    % 12,0,24,0
\definecolor{gruen-dunkel}{cmyk}{0.48,0.09,0.8,0}  % 48,9,80,0
\definecolor{rot-hell}{cmyk}{0,0.26,0.14,0}      % 0,26,14,0
\definecolor{rot-dunkel}{cmyk}{0.19,0.8,0.66,0.08}    % 19,80,66,8
% hell Background
\definecolor{hell1}{cmyk}{0.74,0.04,0.28,0}       % C=74 M=4 Y=28 K=0
\definecolor{hell2}{cmyk}{0.42,0,0.03,0}          % C=42 M=0 Y=3 K=0
\definecolor{hell3}{cmyk}{0.12,0,0.38,0}          % C=12 M=0 Y=38 K=0
% dunkel Background
\definecolor{dunkel1}{cmyk}{0.96,0.59,0.59,0.71}  % C=96 M=59 Y=59 K=71
\definecolor{dunkel2}{cmyk}{0.97,0.73,0,0}        % C=97 M=73 Y=0 K=0
\definecolor{dunkel3}{cmyk}{0.07,1,1,0.2}         % C=7 M=100 Y=100 K=20  

\usepackage[
pdfstartview={FitV}, %Seite so hoch wie Fenster
pdffitwindow=true, %Fenster wird nicht
%automatisch an Seite angepasst
pdfcenterwindow=true, %mittiges Fenster
bookmarksnumbered=true
]{hyperref}

\usepackage{hyperxmp} % XMP-Daten fuer die PDF-Datei

%\hypersetup{%}

% pdf-Lesezeichen
\usepackage[%
openlevel=1, %am Beginn offen...
depth=3, %Tiefe der bookmarks insgesamt
numbered, %Kapitelnr bei bookmarks
]{bookmark}


%----------------------
% Marginalien
\newcommand{\marginlabel}[1]
{\mbox{}\marginpar{\RaggedRight\hspace{0pt}#1}}

\usepackage{nameref}

\usepackage{qrcode}% QR - Code Anwendung: \qrcode[hyperlink,height=4em]{\website}
% Hindergrundgrafik
\usepackage{wallpaper}
\usepackage{eso-pic}


\lstset{%
	basicstyle=\linespread{1}\ttfamily\small,
	floatplacement=htbp,
	captionpos=t,
	abovecaptionskip=.5\baselineskip,
	belowcaptionskip=.5\baselineskip,
	upquote=true,
	showstringspaces=false,
	inputencoding=utf8,
	tabsize=4,
	keywordstyle=\bfseries \color{black},
	commentstyle=\color{OliveGreen},
	stringstyle=\color{BurntOrange},
	breaklines=true,
	breakatwhitespace=true,
}

\lstset{literate={á}{{\'a}}1 {é}{{\'e}}1 {í}{{\'i}}1 {ó}{{\'o}}1 {ú}{{\'u}}1 {Á}{{\'A}}1 {É}{{\'E}}1 {Í}{{\'I}}1 {Ó}{{\'O}}1 {Ú}{{\'U}}1 {à}{{\`a}}1 {è}{{\`e}}1 {ì}{{\`i}}1 {ò}{{\`o}}1 {ù}{{\`u}}1 {À}{{\`A}}1 {È}{{\'E}}1 {Ì}{{\`I}}1 {Ò}{{\`O}}1 {Ù}{{\`U}}1 {ä}{{\"a}}1 {ë}{{\"e}}1 {ï}{{\"i}}1 {ö}{{\"o}}1 {ü}{{\"u}}1 {Ä}{{\"A}}1 {Ë}{{\"E}}1 {Ï}{{\"I}}1 {Ö}{{\"O}}1 {Ü}{{\"U}}1 {â}{{\^a}}1 {ê}{{\^e}}1 {î}{{\^i}}1 {ô}{{\^o}}1 {û}{{\^u}}1 {Â}{{\^A}}1 {Ê}{{\^E}}1 {Î}{{\^I}}1 {Ô}{{\^O}}1 {Û}{{\^U}}1 {œ}{{\oe}}1 {Œ}{{\OE}}1 {æ}{{\ae}}1 {Æ}{{\AE}}1 {ß}{{\ss}}1 {ű}{{\H{u}}}1 {Ű}{{\H{U}}}1 {ő}{{\H{o}}}1 {Ő}{{\H{O}}}1 {ç}{{\c c}}1 {Ç}{{\c C}}1 {ø}{{\o}}1 {å}{{\r a}}1 {Å}{{\r A}}1 {€}{{\EUR}}1 {£}{{\pounds}}1 {~}{{\textasciitilde}}1 {-}{{-}}1 }

% Abb., Tab., Quellcode
\renewcaptionname{ngerman}{\contentsname}{Inhalt}
\renewcaptionname{ngerman}{\listfigurename}{Abbildungen}
\renewcaptionname{ngerman}{\listtablename}{Tabellen}
\renewcaptionname{ngerman}{\figurename}{Abb.}
\renewcaptionname{ngerman}{\tablename}{Tab.}
% names
\let\defaultlstlistingname\lstlistingname\renewcommand\lstlistingname{\iflanguage{ngerman}{Quelltext}{\defaultlstlistingname}}

% captions
\addtokomafont{caption}{\small}
\addtokomafont{captionlabel}{\bfseries\sffamily\small}
\setcapindent{\parindent}


% list of contents
\setcounter{tocdepth}{3}
\setcounter{secnumdepth}{3}




\definecolor{mycolor}{rgb}{0.122, 0.435, 0.698}% Rule colour
\makeatletter
\newcommand{\mybox}[1]{%
	\setbox0=\hbox{#1}%
	\setlength{\@tempdima}{\dimexpr\wd0+13pt}%
	\begin{tcolorbox}[colframe=mycolor,boxrule=0.5pt,arc=4pt,
		left=6pt,right=6pt,top=6pt,bottom=6pt,boxsep=0pt,width=0.95\textwidth]
		#1
	\end{tcolorbox}
}
\makeatother

% Parts list tables
\renewcommand\theadfont{\bfseries}
\newcolumntype{I}{ >{\centering\arraybackslash} m{2cm} }  % part image
\newcolumntype{N}{ >{\centering\arraybackslash} m{2cm} }  % part name
\newcolumntype{Q}{ >{\centering\arraybackslash} m{1cm} }  % ref & menge

% tables
\newcolumntype{L}[1]{>{\raggedright\let\newline\\\arraybackslash\hspace{0cm}}m{#1}}
\newcolumntype{C}[1]{>{\centering\let\newline\\\arraybackslash\hspace{0cm}}m{#1}}
\newcolumntype{R}[1]{>{\raggedleft\let\newline\\\arraybackslash\hspace{0cm}}m{#1}}
\captionsetup[table]{belowskip=.5\baselineskip,aboveskip=.5\baselineskip}

\newcommand\partimg{\includegraphics[width=2cm,height=1.25cm,keepaspectratio]}


\usepackage{blindtext}% \blindtext
% Erzeugung mehrseitiger Tabellen
\usepackage{longtable}

% Eigene Befehle können hier definiert werden
%% Textauszeichnung
% \emph{kursiv}
% \textrm{Antiqua}, \textsf{Grotesk}, \texttt{Maschinenschrift},
% \textmd{normal}, \textbf{breiter}, \textup{aufrecht}, \textsl{geneigt},
% \textit{kursiv}, \textsc{Kapitaelchen}

%% Schriftgroesse
% \tiny{winzig}, \scriptsize{sehr klein}, \footnotesize{klein},
% \small{klein}, \normalsize{normal}, \large{gross}, \Large{groesser},
% \LARGE{ganz gross}, \huge{riesig}, \Huge{gigantisch}

%% eigene Befehle definieren
% Textauszeichnung: \wort{Beispiel}, \fremdwort{prezioes}
\newcommand{\wort}[1]{\emph{#1}}
\newcommand{\fremdwort}[1]{\textsf{#1}}

%% Textabstand:  Verwendung: \abstand{}
\newcommand{\abstand}[1]{\hspace{10em}{#1}}
%% quad, qquad, hspace{10em}, vspace{10em}
%
% Wichtig (Optionale Parameter)
%% Wort Kursiv u. in Farbe
\newcommand{\wichtig}[2][rot6]{\textcolor{#1}{\emph{#2}}}
